\documentclass[11pt]{article}
\usepackage[a4paper,margin=1in]{geometry}
\usepackage{fourier} % Fourier font
\usepackage{xcolor}
\usepackage{tikz}
\usepackage[most]{tcolorbox}
\usepackage{amsthm, amsmath, amssymb}
\usepackage{enumitem}
\usepackage{hyperref}
\usepackage[nameinlink,noabbrev]{cleveref}
\usepackage{titling} 

% Dark mode colors
\definecolor{bgcolor}{HTML}{1E1E1E}
\definecolor{textcolor}{HTML}{FAFAFA}
\definecolor{defcolor}{HTML}{E86873}
\definecolor{thmcolor}{HTML}{0A9396}
\definecolor{lemcolor}{HTML}{94D2BD}
\definecolor{corcolor}{HTML}{9B4AF7}
\definecolor{probcolor}{HTML}{EE9B00}
\definecolor{excolor}{HTML}{21E933}

% Background and text color
\pagecolor{bgcolor}
\color{textcolor}

% No paragraph indentation
\setlength{\parindent}{0pt}
\setlength{\parskip}{0.7em}

% Theorem box styles
\tcbset{
  enhanced,
  colback=bgcolor,
  colframe=thmcolor,
  coltext=white,
  coltitle=white,
  fonttitle=\bfseries,
  boxrule=0.7pt,
  left=1em,
  right=1em,
  top=0.7em,
  bottom=0.7em,
  before skip=10pt,
  after skip=10pt,
}

% Theorem environments with colored boxes
\newtcbtheorem[number within=section]{thm}{Theorem}{
  colframe=thmcolor, colback=thmcolor!15!bgcolor
}{thm} % The 'thm' here is the *prefix* for the label

\newtcbtheorem[number within=section]{defn}{Definition}{
  colframe=defcolor, colback=defcolor!15!bgcolor
}{def} % The 'def' here is the *prefix* for the label

\newtcbtheorem[number within=section]{lem}{Lemma}{
  colframe=lemcolor, colback=lemcolor!15!bgcolor
}{lem}

\newtcbtheorem[number within=section]{cor}{Corollary}{
  colframe=corcolor, colback=corcolor!15!bgcolor
}{cor}

\newtcbtheorem[number within=section]{prob}{Problem}{
  colframe=probcolor, colback=probcolor!15!bgcolor
}{prob}

\newtcbtheorem[number within=section]{ex}{Example}{
  colframe=excolor, colback=excolor!15!bgcolor
}{ex}

% Proof environment 
\renewenvironment{proof}[1][\proofname]{%
  \par\pushQED{\qed}\normalfont\topsep6pt \trivlist
  \item[\hskip\labelsep\itshape #1.]\ignorespaces
}{%
  \popQED\endtrivlist\addvspace{6pt}
}

% Cleveref name formats for tcolorbox environments
\crefname{thm}{theorem}{theorems}
\Crefname{thm}{Theorem}{Theorems}

\crefname{def}{definition}{definitions}
\Crefname{def}{Definition}{Definitions}

\crefname{lem}{lemma}{lemmas}
\Crefname{lem}{Lemma}{Lemmas}

\crefname{cor}{corollary}{corollaries}
\Crefname{cor}{Corollary}{Corollaries}

\crefname{prob}{problem}{problems}
\Crefname{prob}{Problem}{Problems}

\crefname{ex}{example}{examples}
\Crefname{ex}{Example}{Examples}


\title{\huge{Networking Questions}}
\author{}
\date{}

\begin{document}
\maketitle

\section{Answers to MAC addresses}
PC-A MAC: 74d4.dd5b.4efb

PC-B MAC: 9ceb.e822.3a03

S1 MAC: 001c.f97f.0301

S2 MAC: 0023.ac94.d301

PC-A OUI: 74d4dd

PC-A SERIAL: 5b4efb

PC-A VENDOR: Quanta Computer Inc. 

PC-B OUI: 9cebe8

PC-B SERIAL: 223a03 

PC-B VENDOR: BizLink (Kunshan) Co.,Ltd 

S1 VLAN1 OUI: 001cf9

S1 VLAN1 SERIAL: 7f0301

S1 VENDOR: Cisco Systems, Inc 

S2 VLAN1 OUI: 0023ac 

S2 SERIAL: 94d301 

S2 VENDOR: Cisco Systems, Inc

\newpage
\section{Reflection questions}
\emph{Broadcasts at Layer 2.}

A layer 2 broadcast would involve sending data on all ports, 
and since layer 2 is concerned with single communication links, 
the addresses would not cross into different networks?
The address would be FF:FF:FF:FF:FF:FF, in that case.

\emph{Importance of MAC addresses.}

Ip addresses, especially in a network where addresses are automatically
assigned, do not tell you who is who, while a MAC address 
is supposed to uniquely identify the machine.
Additionally if, for instance you were a network administrator in a 
large network, you could fetch the manufacturer behind a network card 
without leaving your chair by fetching the MAC address. 


\end{document}
