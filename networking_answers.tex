\input{preamble.tex}

\title{\huge{Networking Questions}}
\author{}
\date{}

\begin{document}
\maketitle

\section{Answers to MAC addresses}
PC-A MAC: 74d4.dd5b.4efb

PC-B MAC: 9ceb.e822.3a03

S1 MAC: 001c.f97f.0301

S2 MAC: 0023.ac94.d301

PC-A OUI: 74d4dd

PC-A SERIAL: 5b4efb

PC-A VENDOR: Quanta Computer Inc. 

PC-B OUI: 9cebe8

PC-B SERIAL: 223a03 

PC-B VENDOR: BizLink (Kunshan) Co.,Ltd 

S1 VLAN1 OUI: 001cf9

S1 VLAN1 SERIAL: 7f0301

S1 VENDOR: Cisco Systems, Inc 

S2 VLAN1 OUI: 0023ac 

S2 SERIAL: 94d301 

S2 VENDOR: Cisco Systems, Inc

\newpage
\section{Reflection questions}
\emph{Broadcasts at Layer 2.}

A layer 2 broadcast would involve sending data on all ports, 
and since layer 2 is concerned with single communication links, 
the addresses would not cross into different networks?
The address would be FF:FF:FF:FF:FF:FF, in that case.

\emph{Importance of MAC addresses.}

Ip addresses, especially in a network where addresses are automatically
assigned, do not tell you who is who, while a MAC address 
is supposed to uniquely identify the machine.
Additionally if, for instance you were a network administrator in a 
large network, you could fetch the manufacturer behind a network card 
without leaving your chair by fetching the MAC address. 


\end{document}
